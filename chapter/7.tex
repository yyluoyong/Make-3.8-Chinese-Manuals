\chapter{Makefile的条件执行}
条件语句可以根据一个变量的值来控制make执行或者忽略Makefile的特定部分。条件语
句可以是两个不同变量、或者变量和常量值的比较。要注意的是:条件语句只能用于控
制make实际执行的makefile文件部分,它不能控制规则的shell命令执行过程。Makefile
中使用条件控制可以做到处理的灵活性和高效性。

\section{一个例子}
首先我们来看一个使用条件判断的Makefile例子;对变量“CC”进行判断,其值如果是
“gcc”那么在程序连接时使用库“libgnu.so”或者“libgnu.a”,否则不链接任何库。
Makefile中的条件判断部分如下:

\begin{Verbatim}[]
……
libs_for_gcc = -lgnu
normal_libs =

……

foo: $(objects)

ifeq ($(CC),gcc)
    $(CC) -o foo $(objects) $(libs_for_gcc)
else
    $(CC) -o foo $(objects) $(normal_libs)
endif
……
\end{Verbatim}

例子中,条件语句中使用到了三个关键字:“ifeq”、“else”和“endif”。其中:

1. “ifeq”表示条件语句的开始,并指定了一个比较条件(相等)。之后是用圆括号括包
围的、使用逗号“,”分割的两个参数,和关键字“ifeq”用空格分开。参数中的变量引用在
进行变量值比较时被展开。“ifeq”之后就是当条件满足make需要执行的,条件不满足时
忽略。

2. “else”之后就是当条件不满足时的执行部分。不是所有的条件语句都需要此部分。

3. “endif”表示一个条件语句的结束,任何一个条件表达式都必须以“endif”结束。

通过上边的例子我们可以了解到。Makefile中,条件表达式工作于文本级别(条件判断
处理为文本级别的处理过程),条件的解析是由make来完成的。make是在读取并解析
Makefile时根据条件表达式忽略条件表达式中的某一个文本行,解析完成后保留的只有
表达式满足条件所需要执行的文本行。上例,make处理条件的过程:

当变量“CC”的值为“gcc”时,整个条件表达式等效于:

\begin{Verbatim}[]
foo: $(objects)
    $(CC) -o foo $(objects) $(libs_for_gcc)
\end{Verbatim}

当变量“CC”值不等于“gcc”时等效于:
\begin{Verbatim}[]
foo: $(objects)
    $(CC) -o foo $(objects) $(normal_libs)
\end{Verbatim}

上面的例子,一种更简洁实现方式:
\begin{Verbatim}[]
libs_for_gcc = -lgnu
normal_libs =

ifeq ($(CC),gcc)
libs=$(libs_for_gcc)
else
libs=$(normal_libs)
endif

foo: $(objects)
    $(CC) -o foo $(objects) $(libs)
\end{Verbatim}

\section{条件判断的基本语法}
一个简单的不包含“else”分支的条件判断语句的语法格式为:
\begin{Verbatim}[]
CONDITIONAL-DIRECTIVE
TEXT-IF-TRUE
endif
\end{Verbatim}
\noindent 表达式中“TEXT-IF-TRUE”可以是若干任何文本行,当条件为真时它就将被
make作为需要执行的一部分。当条件为假时,不作为需要执行的一部分。

包含“else”的复杂一点的语法格式为:
\begin{Verbatim}[]
CONDITIONAL-DIRECTIVE
TEXT-IF-TRUE
else
TEXT-IF-FALSE
endif
\end{Verbatim}
\noindent 表示了如果条件为真,则将“TEXT-IF-TRUE”作为执行Makefile的一部分,否
则将“TEXT-IF-FALSE”作为执行的Makefile的一部分。和“TEXT-IF-TRUE”一样,
“TEXT-IF-FALSE”可以是若干任何文本行。

条件判断语句中“CONDITIONAL-DIRECTIVE”对于上边的两种格式都是同样的。可以是以下
四种用于测试不同条件的关键字。

\subsection{关键字“ifeq”}
此关键字用来判断参数是否相等,格式如下:
\begin{Verbatim}[]
'ifeq (ARG1, ARG2)'
'ifeq 'ARG1' 'ARG2''
'ifeq "ARG1" "ARG2"'
'ifeq "ARG1" 'ARG2''
'ifeq 'ARG1' "ARG2"'
\end{Verbatim}

替换展开“ARG1”和“ARG1”后,对它们的值进行比较。如果相同则(条件为真)将
“TEXT-IF-TRUE”作为make要执行的一部分,否则将“TEXT-IF-FALSE”作为make要执行的一
部分(上边的第二种格式)。

通常我们会使用它来判断一个变量的值是否为空(不是任何字符)。参数值可能是通过
引用变量或者函数得到的,因而在展开过程中可能造成参数值中包含空字符(空格
等)。一般在这种情况时我们使用make的“strip”函数来对它变量的值进行处理,去掉其
中的空字符。格式为:

\begin{Verbatim}[]
ifeq ($(strip $(foo)),)
TEXT-IF-EMPTY
endif
\end{Verbatim}
这样,即就是在“\$(foo)”中存在若干前导和结尾空格,“TEXT-IF-EMPTY”也会被作为
Makefile需要执行的一部分。

\subsection{关键字“ifneq”}
此关键字是用来判断参数是否不相等,格式为:
\begin{Verbatim}[]
'ifneq (ARG1, ARG2)'
'ifneq 'ARG1' 'ARG2''
'ifneq "ARG1" "ARG2"'
'ifneq "ARG1" 'ARG2''
'ifneq 'ARG1' "ARG2"'
\end{Verbatim}
关键字“ifneq”实现的条件判断语句和“ifeq”相反。首先替换并展开“ARG1”和“ARG1”,对
它们的值进行比较。如果不相同(条件为真)则将“TEXT-IF-TRUE”作为make要执行的一
部分,否则将“TEXT-IF-FALSE”作为make要执行的一部分。

\subsection{关键字“ifdef”}
关键字“ifdef”用来判断一个变量是否已经定义。格式为:
\begin{Verbatim}[]
'ifdef VARIABLE-NAME'
\end{Verbatim}

如果变量“VAEIABLE\_NAME”的值非空(在Makefile中没有定义的变量的值为空),那么
表达式为真,将“TEXT-IF-TRUE”作为make要执行的一部分。否则,表达式为假,如果存
在“TEXT-IF-FALSE”,就将它作为make要执行一部分。当一个变量没有被定义时,它的值
为空。“VARIABLE-NAME”可以是变量或者函数的引用。

对于“ifdef”需要说明的是:\textbf{ifdef只是测试一个变量是否有值,不会对变量进
行替换展开来判断变量的值是否为空}。对于变量“VARIABLE-NAME”,除了
“VARIABLE-NAME=”这种情况以外,使用其它方式对它的定义都会使“ifdef”返回真。就是
说,即使我们通过其它方式(比如,定义它的值引用了其它的变量)给它赋了一个空
值,“ifdef”也会返回真。我们来看一个例子:

例1:
\begin{Verbatim}[]
bar =
foo = $(bar)
ifdef foo
frobozz = yes
else
frobozz = no
endif
\end{Verbatim}

例2:
\begin{Verbatim}[]
foo =
ifdef foo
frobozz = yes
else
frobozz = no
endif
\end{Verbatim}

例1中的结果是:“frobozz = yes”;而例2的结果是:“frobozz = no”。其原因就是在例
1中,变量“foo”的定义是“foo = \$(bar)”。虽然变量“bar”的值为空,但是“ifdef”判断
的结果是真。因此当我们需要判断一个变量的值是否为空的情况时,需要使用“ifeq”
(或者“ifneq”)而不是“ifdef”。可参考前两个小节的内容。

\subsection{关键字“ifndef”}
关键字“ifndef”实现的功能和“ifdef”相反。格式为:
\begin{Verbatim}[]
'ifdef VARIABLE-NAME'
\end{Verbatim}
这个就不详细讨论了,它的功能就是实现了和“ifdef”相反的条件判断。

在“CONDITIONAL-DIRECTIVE”这一行上,可以以若干个空格开始,make处理时会被忽略这
些空格。但不能以[Tab]字符做为开始(不然就被认为是命令)。条件判断语句中,在除
关键字(包括“endif”)之前、条件表达式参数中之外的其他任何地方都可以使用多个空
格或[Tab]字符,它不会影响条件判断语句的功能。同样行尾也可以使用注释(“\#”开始
直到一行的结束)。“else”和“endif”也是条件判断语句的一部分。在书写时它们都是没
有任何参数的,可以以多个空格开始(同样不能以[Tab]字符开始)多个空格或[Tab]字
符结束。行尾同样可以有注释内容。

在make读取makefile文件时计算表达式的值,并根据表达式的值决定判断语句中那一部
分被作为此Makefile所要执行的内容(选择符合条件的语句)。因此在条件表达式中不
能使用自动化变量,自动化变量在规则命令执行时才有效。更不能将一个完整的条件判
断语句分写在两个不同的makefile文件中,在一个makefile文件使用指示符“include”包
含另外一个makefile文件。

\section{标记测试的条件语句}
我们可以使用条件判断语句、变量“MAKEFLAGS”和函数“findstring”,实现对make命令行
选项的测试。看一个例子:
\begin{Verbatim}[]
archive.a: ...
ifneq (,$(findstring t,$(MAKEFLAGS)))
    +touch archive.a
    +ranlib -t archive.a
else
    ranlib archive.a
endif
\end{Verbatim}

这个条件语句判断make的命令行参数中是否包含“-t”(用来更新目标文件的时间戳)。
根据命令行参数情况完成对“archive.a”执行不同的操作。命令行前的“+”的意思是告诉
make,即使make使用了“-t”参数,“+”之后的命令都需要被执行。
