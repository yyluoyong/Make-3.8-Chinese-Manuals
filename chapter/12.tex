\chapter{GNU make的特点}
截至本章为止,所有关于GNU make使用的讨论全部结束。相信大家也能够独立、熟练的
书写Makefile,并能够使用GNU 的make来管理自己的工程。

本章是GNU make特点的总结。主要是和其它版本make得比较。这些特征都是以4.2 BSD
中的make的为基准。当需要书写可移植到不同的类UNIX系统上的Makefile时,应避免使
用GNU 版本make自身的一些特征。

\section{源自System v的特点}
下面所罗列的这些是来自System V版本make的一些特点:

\begin{dinglist}{226}
\itemsep=4pt

\item 变量“VPATH”及它的含义(参考 4.5 目录搜索 一节)。System V版本的make支
    持,但没有得到验证。4.3 BSD 的make支持(据说是对System V的make这一功能的
    仿照)。

\item 可包含其它makefile文件(参考 3.3 包含其它makefile文件 一节)。使用指示
    符“include”可同时包含多个文件是GNU的扩展。

\item 可使用系统环境变量。参考 6.9 系统环境变量 一节。

\item 变量“MAKEFLAGS”。用于make递归调用时传递命令行选项。参考 5.6 make的递归
    调用 一节。

\item  make静态库(文档文件)是“\$\%”可代表静态库成员名。参考 10.5.3 自动化变
    量 一节。

\item 对自动变量“\$@”、“\$*”、“\$\verb"<"”、“\$\%”和“\$?”进行了扩展,支持
    “\$(@F)”和
    “\$(@D)”等形式的自动环变量。并以此对自动变量“\$\verb"^"”进行了扩展。参考 10.5.3 自
    动化变量 一节。

\item 变量引用。参考 6.1变量的引用 一节。

\item 支持使用命令行选项“-b”和“-m”来兼容其它版本的make。System V 的make中,这
    些选项有实际含义。

\item 使用变量“MAKE”执行的递归调用。可支持“-n”、“-q”和“-t”向子make进程的传
    递。参考  5.6 make的递归调用 一节。

\item 支持后缀“.a”(参考 11.4 静态库的后缀规则 一节)。不过这个特点在GNU make
    的新版本中已经被模式规则所取代。

\item 保持Makefile规则命令行的书写格式,只是去掉了初始空字符。执行命令的回显
    保持Makefile中的书写格式不变。

\end{dinglist}

\section{源自其他版本的特点}

下面的特点来自于其它版本的make,但每一个特征来自由哪个版本的make不太清楚:

\begin{dinglist}{226}
\itemsep=4pt

\item 模式规则使用模式字符“\%”。目前在多个不同版本的make中都有使用“\%”。但具
    体是那个版本的make提出它不甚清楚。关于模式规则参考  10.5 模式规则 一节。

\item 规则链以及隐含的中间过程文件。这个特点首次是在Stu Feldman 的make版本中
    实现,并用于AT\&T第八版的Unix研究中。后来AT\&T贝尔实验室的Andrew Hume 在它
    的mk程序中应用(这里称为“传递闭合”)。但不清楚这个特征是对它们的继承还是
    GNU自己的重新实现。参考 10.4 隐含规则链 一节。

\item 包含规则所有依赖文件列表的自动化变量“\$\verb"^"”。可以确定这不是GNU创造
    的,但
    具
    体是哪个版本的make创造也不清楚。参考 10.5.3 自动化变量 一节。

\item 命令行的“what if”选项(GNU make的“-W”选项)据说是Andrew Hume 在mk中首次
    提出的的。参考 9.7 make的命令行选项 一节。

\item  并发执行的观点,在其它多种版本的make中都有支持。但System V 和BSD 没有
    实现此功能。参考 5.3 并发执行命令 一节。

\item 变量的模式替换引用来自SunOS 4。参考 6.3 变量的高级用法 一节。GNU make
    中,这个功能是在SunOS 4实现之前由函数“patsubst”提供。同一功能的两种实现,
    在两个版本的make中,难以确定是哪一个最早提出这个概念。

\item 命令行之前使用“+”字符,它有特殊的含义(参考 9.3 替代命令的执行 一节)。
    这种做法是由IEEE Standard 1003.2-1992 (POSIX.2)定义的。

\item 变量值追加“+=”的语法来自于SunOS 4 版本的make。参考 6.6 追加变量值 一
    节。

\item 静态库成员列表作为目标的语法“ARCHIVE(MEM1 MEM2...)”源自SunOS 4 make。参
    考 11.1 库成员作为目标 一节。

\item 使用“-include”包括多个其它的makefile文件,当所包含的文件不存在时不出
    错。这个特征源自SunOS 4 版本的make。(但SunOS 4版本的make不能使用单指示符
    同时包含多个文件) 。GNU make的这个特征和SGI make 的指示符“sinclude”相
    同。GNU make也支持“sinclude”。

\end{dinglist}

\section{GNU make自身的特点}

以下特点是由GNU make本身的:

\begin{dinglist}{226}
\itemsep=4pt

\item 命令行选项“-v”或“--version”打印make的版本和版权信息。

\item 使用“-h”或“--help”列出make支持的所有命令行选项。

\item 直接展开式变量。参考 6.2 两种变量定义 一节。

\item 变量“MAKE”支持make递归调用时命令行选项的传递。参考  5.6 make的递归调用
    一节。

\item  命令选项“-C”或“--directory”改变make执行的工作目录。参考 9.7 make的命令
    行选项 一节。

\item 支持多行变量的定义。参考 6.8 定义多行变量 一节。

\item 使用特殊目标“.PHONY”声明伪目标。AT\&T 贝尔实验室Andrew Hume 使用不同的
    语
    法在它的mk程序中也实现了该功能。两者几乎在同时支持。参考 4.6 Makefile伪目
    标 一节。

\item 支持过个文本处理函数。参考 8.2 文本处理函数 一节。

\item 支持使用“-o”或者“--old-file”选项指定一个文件是未修改文件(告诉make不需
    要考虑这个文件的时间戳)。参考 9.4 防止特定文件重建 一节。

\item 条件执行。众多其它版本的make也支持;它似乎是对c语言预处理程序和宏语言的
    自然扩展,不能算是一个全新的概念。参考 第七章 Makefile的条件判断 一章。

\item 指定包含makefile文件的搜寻路径。参考 3.3 包含其它makefile文件 一节。

\item 环境变量“MAKEFILES”指定需要默认读取的makefile文件。参考 3.4 变量
    MAKEFILES 一节。

\item 去除文件名中的“./”,因此“./file”和“file”等价。

\item 支持在规则依赖中使用“-lNAME”来指定连接库。参考 4.5 目录搜索 一节。

\item 允许后缀规则中后缀名可以是任何字符串(参考 10.7 后缀规则 一节)。其它版
    本make中后缀必须以“.”开始,并且后缀中不能包含斜杠“/”。

\item 内嵌变量“MAKELEVEL”,这个变量被用来跟踪make递归调用过程调用深度。 参考
    5.6 make的递归执行 一节。

\item 内嵌变量“MAKECMDGOALS”,这个变量代表了make执行的终极目标。8.2 指定终极
    目标 一节。

\item 指定静态模式规则。参考 4.12 静态模式 一节。

\item 支持选择性搜索关键字“vpath”。参考 4.5 目录搜索 一节。

\item 支持计算的变量引用。参考 6.3 变量的高级用法 一节。

\item 支持自动重建makefile文件(参考 3.7 makefile文件的重建 一节)。

\item 多个新的内嵌隐含规则。

\item 内嵌变量“MAKE\_VERSION”代表当前的make版本。

\end{dinglist}
